% Options for packages loaded elsewhere
\PassOptionsToPackage{unicode}{hyperref}
\PassOptionsToPackage{hyphens}{url}
%
\documentclass[
]{article}
\usepackage{amsmath,amssymb}
\usepackage{iftex}
\ifPDFTeX
  \usepackage[T1]{fontenc}
  \usepackage[utf8]{inputenc}
  \usepackage{textcomp} % provide euro and other symbols
\else % if luatex or xetex
  \usepackage{unicode-math} % this also loads fontspec
  \defaultfontfeatures{Scale=MatchLowercase}
  \defaultfontfeatures[\rmfamily]{Ligatures=TeX,Scale=1}
\fi
\usepackage{lmodern}
\ifPDFTeX\else
  % xetex/luatex font selection
\fi
% Use upquote if available, for straight quotes in verbatim environments
\IfFileExists{upquote.sty}{\usepackage{upquote}}{}
\IfFileExists{microtype.sty}{% use microtype if available
  \usepackage[]{microtype}
  \UseMicrotypeSet[protrusion]{basicmath} % disable protrusion for tt fonts
}{}
\makeatletter
\@ifundefined{KOMAClassName}{% if non-KOMA class
  \IfFileExists{parskip.sty}{%
    \usepackage{parskip}
  }{% else
    \setlength{\parindent}{0pt}
    \setlength{\parskip}{6pt plus 2pt minus 1pt}}
}{% if KOMA class
  \KOMAoptions{parskip=half}}
\makeatother
\usepackage{xcolor}
\usepackage[margin=1in]{geometry}
\usepackage{longtable,booktabs,array}
\usepackage{calc} % for calculating minipage widths
% Correct order of tables after \paragraph or \subparagraph
\usepackage{etoolbox}
\makeatletter
\patchcmd\longtable{\par}{\if@noskipsec\mbox{}\fi\par}{}{}
\makeatother
% Allow footnotes in longtable head/foot
\IfFileExists{footnotehyper.sty}{\usepackage{footnotehyper}}{\usepackage{footnote}}
\makesavenoteenv{longtable}
\usepackage{graphicx}
\makeatletter
\def\maxwidth{\ifdim\Gin@nat@width>\linewidth\linewidth\else\Gin@nat@width\fi}
\def\maxheight{\ifdim\Gin@nat@height>\textheight\textheight\else\Gin@nat@height\fi}
\makeatother
% Scale images if necessary, so that they will not overflow the page
% margins by default, and it is still possible to overwrite the defaults
% using explicit options in \includegraphics[width, height, ...]{}
\setkeys{Gin}{width=\maxwidth,height=\maxheight,keepaspectratio}
% Set default figure placement to htbp
\makeatletter
\def\fps@figure{htbp}
\makeatother
\setlength{\emergencystretch}{3em} % prevent overfull lines
\providecommand{\tightlist}{%
  \setlength{\itemsep}{0pt}\setlength{\parskip}{0pt}}
\setcounter{secnumdepth}{-\maxdimen} % remove section numbering
\renewcommand{\rmdefault}{bch}
\usepackage{tikz}
\usetikzlibrary{positioning}
\ifLuaTeX
  \usepackage{selnolig}  % disable illegal ligatures
\fi
\IfFileExists{bookmark.sty}{\usepackage{bookmark}}{\usepackage{hyperref}}
\IfFileExists{xurl.sty}{\usepackage{xurl}}{} % add URL line breaks if available
\urlstyle{same}
\hypersetup{
  pdftitle={COMPARAÇÃO DAS TÉCNICAS DE AGRUPAMENTO:},
  pdfauthor={Fabiano F. dos Santos},
  hidelinks,
  pdfcreator={LaTeX via pandoc}}

\title{COMPARAÇÃO DAS TÉCNICAS DE AGRUPAMENTO:}
\usepackage{etoolbox}
\makeatletter
\providecommand{\subtitle}[1]{% add subtitle to \maketitle
  \apptocmd{\@title}{\par {\large #1 \par}}{}{}
}
\makeatother
\subtitle{ESTUDO DE CASO EM DADOS DE VACINAÇÃO E MORTALIDADE INFANTIL
ENTRE OS ANOS DE 2011 A 2021.}
\author{Fabiano F. dos Santos}
\date{CAMPINA GRANDE DD/MM/AA}

\begin{document}
\maketitle

\hypertarget{introduuxe7uxe2o}{%
\subsubsection{INTRODUÇÂO}\label{introduuxe7uxe2o}}

\#\#\#\#ÁNALISE DE AGRUPAMENTOS

--

\begin{itemize}
\tightlist
\item
  ML (Machine Learning) é um subcampo da inteligência artificial (IA)
  que se concentra no desenvolvimento de algoritmos e modelos
  computacionais;
\end{itemize}

--

\begin{itemize}
\tightlist
\item
  Deep learning com subárea de ML onde se faz uso de redes neurais
  artificiais com múltiplas camadas com o intuito de aprender cada vez
  mais dados cada vez mais complexos e contexto do trabalho em questão
  se faz uso das técnicas de aprendizado não supervisionado onde apenas
  especificamos o que se feito automaticamente independentemente do
  programa.
\end{itemize}

--

\begin{itemize}
\tightlist
\item
  Entre as estratégias de ML temos três tipos de aprendizados, incluindo
  aprendizado supervisionado, aprendizado não supervisionado e
  aprendizado por reforço;
\end{itemize}

--

\begin{itemize}
\tightlist
\item
  No contexto do trabalho em questão se faz uso das técnicas de
  aprendizado não supervisionado onde apenas especificamos o que se
  feito automaticamente independentemente do programa.
\end{itemize}

\begin{center}\rule{0.5\linewidth}{0.5pt}\end{center}

\#\#\#MOTIVAÇÕES

--

Recordamos que a necessidade de agrupar é algo intuitivo e inerente ao
ser humano se levarmos em consideração que para sobreviver e compreender
os fenômenos precisamos ordenar os objetos segundo algum critério e
posteriormente classificá-los.

--

\begin{itemize}
\item
  Em áreas das ciências como a biologia, na antiguidade estudos sobre a
  taxonomia que dizem respeito à classificação dos seres vivos, é
  importante enfatizar que um dos primeiros a sugerir um modelo
  taxonômico foi o filósofo grego Aristóteles (384-322 a.C.);
\end{itemize}

--

\begin{itemize}
\item
  Podemos citar casos da vida cotidiana, por exemplo, quando uma criança
  na escola ao receber um conjunto de lápis para pintar um desenho,
  seleciona as principais cores de seu gosto, para daí começar a pintar;
\end{itemize}

--

\begin{itemize}
\item
  Por exemplo, quando um assistente administrativo de uma empresa está
  organizando documentos em determinado setor, é normalmente utilizado
  sistemas computacionais onde determinadas pastas se localizam
  separadas e listadas pelos nomes para otimizar o tempo e organização.
\end{itemize}

\begin{center}\rule{0.5\linewidth}{0.5pt}\end{center}

\hypertarget{objetivos}{%
\subsubsection{Objetivos}\label{objetivos}}

--

\begin{itemize}
\tightlist
\item
  Objetivo Geral:
\end{itemize}

--

\begin{itemize}
\tightlist
\item
  Análise, discussão e comparação acerca das técnicas de clusterização
  hirárquicas e não hierárquicas.
\end{itemize}

--

\begin{itemize}
\tightlist
\item
  Objetivos Específicos:
\end{itemize}

--

\begin{itemize}
\tightlist
\item
  tipos de conversões de variáveis da matriz de dados;
\end{itemize}

--

\begin{itemize}
\tightlist
\item
  as medidas, sendo essas de similaridade ou dissimilaridade;
\end{itemize}

--

\begin{itemize}
\tightlist
\item
  Abordar as principais técnicas de agrupamentos hierárquicas e
  não-hierárquicas;
\end{itemize}

--

\begin{itemize}
\tightlist
\item
  Escolha do número ideal de K partições e posteriores requisitos da
  qualidade do agrupamento.
\end{itemize}

--

\begin{itemize}
\tightlist
\item
  análise exploratória dos dados; seleção de variáveis;
\end{itemize}

--

\begin{itemize}
\tightlist
\item
  interpretações posteriores dos resultados.
\end{itemize}

--

\begin{center}\rule{0.5\linewidth}{0.5pt}\end{center}

\#\#\#REVISÃO BIBLIOGRÁFICA \#\#\#\#MATRIZ ORIGINAL E PROXIMIDADE --

Devemos destacar os agrupamentos dos quais partirmos da matriz
multivariada expressa na forma:

--

\[X = \begin{bmatrix}
X_{11} & X_{12} & \ldots & X_{1j} \\
X_{21} & X_{22} & \ldots & X_{2j} \\
\vdots & \vdots & \ddots & \vdots \\
X_{i1} & X_{i2} & \ldots & X_{ij} \\
\end{bmatrix}_{n \times p},\]

conhecida como matriz multivariada/original, possui dimensão
\({n \times p}\), onde \(n\) é o número de observações e \(p\) é número
de variáveis, lembramos que \(i = 1, \ldots , n\) e
\(j = 1 , \ldots , p.\)

--

As variáveis podem ser categóricas ou métricas, podemos derivar e obter
a matriz de proximidades ou distâncias, na forma:

\begin{longtable}[]{@{}
  >{\raggedright\arraybackslash}p{(\columnwidth - 0\tabcolsep) * \real{0.0417}}@{}}
\toprule\noalign{}
\endhead
\bottomrule\noalign{}
\endlastfoot
\(D = \begin{bmatrix}
d_{11} & d_{21} & d_{31} & \ldots & d_{n1} \\
d_{12} & d_{22} &  d_{32} & \ldots & d_{n2} \\
d_{13} & d_{23} & d_{33}  & \ldots  & d_{n3} \\
\vdots & \vdots & \vdots  & \ddots & \vdots \\
d_{1n}& d_{2n} & d_{3n} &\ldots &  d_{nn}\\
\end{bmatrix}_{n \times n}\) \\
\end{longtable}

\hypertarget{etapas-iniciais-da-clusterizauxe7uxe3o}{%
\paragraph{Etapas iniciais da
clusterização}\label{etapas-iniciais-da-clusterizauxe7uxe3o}}

Habitualmente seguimos alguns critérios a priori para daí agrupar, sendo
eles:

--

\begin{itemize}
\item
  A metodologia da clusterização, seleção de objetos, seleção de
  variáveis, transformação de variáveis, seleção da medida de semelhança
  ou dissemelhança, método de formação de clusters a aplicar, discussão
  e apresentação dos resultados.
\end{itemize}

--

Em relação ao tipo de transformação (QUINTAL, 2005) menciona alguns
tipos de transformação nos dados usamos sejam funções de transformação e
funções de estandardização como podemos observar:

--

\begin{itemize}
\item
  \(Z_{ij} = \sqrt{X_{ij}}\), dados com variância alta e
  consequentemente queremos estabilizar torando mais próxima de
  distribuição normal;
\end{itemize}

--

\begin{itemize}
\item
  \(Z_{ij} = \log{X_{ij}}\), temos o comportamento da variação de forma
  exponencial, podemos linearizar relações exponencias, assim como
  controlar a variância e aproximar de uma distribuição normal;
\end{itemize}

--

\begin{itemize}
\item
  \(Z_{ij} = \frac{X_{ij}-\bar{X_j}}{S_j}\), se faz útil quando queremos
  eliminar a influência de variáveis em outras escalas de valores;
\end{itemize}

--

\begin{itemize}
\item
  \(Z_{ij} = \frac{X_{ij}-Rmin_j}{Rmáx_j-Rmín_j}\) os valores podem
  variar em -1 a 1, assim como a média e o valor do desvio padrão podem
  variar diferentemente da aplicação da fórmula de padronização z-score.
\end{itemize}

\begin{center}\rule{0.5\linewidth}{0.5pt}\end{center}

\#\#\#\#Medidas de (des)semelhanças

Uma forma de mensurar o grau de parentesco entre dois objetos, ou que
quantifique o quanto eles estão próximos ou afastados. Esta medida,
segundo (BUSSAD et al., 1990) será chamada de coeficiente de parecença.

--

\begin{itemize}
\item
  \({d_{ij} = \left [\sum_{k = 1}^{p}(X_{ik}-X_{jk})^2 \right ]^{1/2}}\),
  a considerada a métrica a mais elementar dentre todas as outras
  medidas de distância, mede diretamente de um ponto ao outro os
  elementos;
\end{itemize}

--

\begin{itemize}
\item
  \(e_{ij} = \left [ \sum_{k = 1}^{p}\frac{(X_{ik}-X_{jk})^2}{n} \right ]^{1/2}\),
  indicada por \(e_{pq}\) , tem como principal vantagem a falta de
  valores na matriz multivariada;
\end{itemize}

--

\begin{itemize}
\item
  \(d_{ij} = \sum_{k=1}^{p}\frac{|X_{ik}-X_{jk}|}{|X_{ik}|+|X_{jk}|}\),
  se faz útil quando contemos outliers na distribuição (TIMM,
  2002).Também utilizada em casos em que temos valores próximos de zero
  entre os elementos;
\end{itemize}

--

\begin{itemize}
\item
  \(d_{rs}^2 = (X_r- X_s)^TS^{-1}(X_r - X_s)\), pode fornecer redução da
  dependência das unidades de medição, além ser uma escolha viável em
  dados com alta correlação;
\end{itemize}

--

\begin{itemize}
\item
  \(d_{ij} = \sum_{k = 1}^{p} \left | X_{ik}-X_{jk} \right |\),
  evidencia que os valores atípicos são consideravelmente menos afetados
  por valores do extremo da distribuição.
\end{itemize}

\begin{center}\rule{0.5\linewidth}{0.5pt}\end{center}

Antes de tudo, quando nos interessamos em calcular qualquer semelhança
entre dois elementos, devemos ter em mente que estamos trabalhando com
variáveis do tipo categóricas, nominais e ordinais, inicialmente vamos
particionar os K níveis/categorias das mesmas em (K-1) classes de
variáveis binárias as quais só podem assumir valores 0 e 1, sendo 0 a
falta de um determinada característica de interesse e 1 a presença.

--

\begin{tabular}{ l r r r }
  \hline & \multicolumn{3}{c}{Notas}\\
  \cline{2 - 4} % linha horizontal entre as colunas
% 2 e 4
  \multirow[c]{-2}{*}{Nome} & Prova 1 & Prova 2 & Média\\
  \hline
João Silva & 5{,}0 & 6{,}0 & 5{,}5\\
Maria Oliveira & 7{,}0 & 6{,}0 & 6{,}5\\
Isabela Medeiros & 8{,}6 & 9{,}4 & 9{,}0\\
  \hline
\end{tabular}

\hypertarget{section}{%
\subsection{--}\label{section}}

\begin{figure}

{\centering \includegraphics{slide-pibic_files/figure-latex/unnamed-chunk-1-1} 

}

\caption{Figura 1: Gráfico de barra, números e percentuais de mortalidade}\label{fig:unnamed-chunk-1}
\end{figure}

\begin{center}\rule{0.5\linewidth}{0.5pt}\end{center}

\begin{figure}

{\centering \includegraphics{slide-pibic_files/figure-latex/unnamed-chunk-2-1} 

}

\caption{Figura 2: Gráfico de barra, números e percentuais de imunização}\label{fig:unnamed-chunk-2}
\end{figure}

\begin{center}\rule{0.5\linewidth}{0.5pt}\end{center}

\begin{figure}

{\centering \includegraphics{slide-pibic_files/figure-latex/unnamed-chunk-3-1} 

}

\caption{Figura 3: Diagrama de caixa, distribuição da variável óbitos}\label{fig:unnamed-chunk-3}
\end{figure}

\begin{center}\rule{0.5\linewidth}{0.5pt}\end{center}

\begin{figure}

{\centering \includegraphics{slide-pibic_files/figure-latex/unnamed-chunk-4-1} 

}

\caption{Figura 4: Diagrama de caixa, distribuição da variável Imunização}\label{fig:unnamed-chunk-4}
\end{figure}

\begin{center}\rule{0.5\linewidth}{0.5pt}\end{center}

\begin{figure}

{\centering \includegraphics{slide-pibic_files/figure-latex/unnamed-chunk-5-1} 

}

\caption{Figura 5: Histogramas, visualização das densidades}\label{fig:unnamed-chunk-5}
\end{figure}

\begin{center}\rule{0.5\linewidth}{0.5pt}\end{center}

\begin{figure}

{\centering \includegraphics{slide-pibic_files/figure-latex/unnamed-chunk-6-1} 

}

\caption{Figura 6:Método do coeficiente silhueta médio}\label{fig:unnamed-chunk-6}
\end{figure}

\begin{center}\rule{0.5\linewidth}{0.5pt}\end{center}

\begin{figure}

{\centering \includegraphics{slide-pibic_files/figure-latex/unnamed-chunk-7-1} 

}

\caption{Figura 7: Método do cotovelo}\label{fig:unnamed-chunk-7}
\end{figure}

\end{document}
